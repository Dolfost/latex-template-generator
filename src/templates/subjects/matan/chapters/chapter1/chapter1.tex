\documentclass[\mainDocument]{subfiles}

\begin{document}
	\nocite{*}
	\chapter{Знайти суму ряду за означенням}
	\begin{gather}
		\sum_{n=1}^\infty\frac{7}{49n^2+7n-12} \label{eq:1}
	\end{gather}

	\solving
	Необхідна умова збіжності виконується:
	\begin{gather}
		\lim_{n\to\infty} \frac{7}{49n^2+7n-12} \equiv 0
	\end{gather}
	Розкладемо \(f(x) = \frac{7}{49n^2+7n-12}\) на елементарні дроби:
	\ldots
	\ansver
	Сума ряду \ref{eq:1} рівна \(\frac{49}{9}\).

	\chapter{Дослідити на збіжність}
	Дослідити на збіжність знакододатний числовий ряд за ознаками порівняння
	\begin{gather}
		\sum_{n=3}^\infty \frac{n+5}{(n^2)(n+2)} \label{eq:2}
	\end{gather}

	\begin{thm}[Гранична ознака порівняння]
		\label{dfn:zbiz}
		Нехай \(\sum_{n=1}^\infty u_n\) і \(\sum_{n=1}^\infty v_n\) -- ряди з додатними членами. Якщо існує скінченна, відмінна від нуля, границя
		\begin{gather}
			\lim_{n\to\infty}\frac{u_n}{v_n}=k\ (0<k<\infty)
		\end{gather}
		то вказані ряди одночасно збіжні або розбіжні.
	\end{thm}
	\solving
	\ldots
	\ansver
	Ряд \ref{eq:2} збіжний.
\end{document}
