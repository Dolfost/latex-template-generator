\documentclass[\mainDocument]{subfiles}

% Постановка задачі

\begin{document}
\BeforeChpater{}

\chapter{Постановка задачі}
Вміст цієї секції взятий з~\cite{pristavka_matstat}.\\
\bigtext{Тема:} \worktheme.\\
\bigtext{Мета:} 
\begin{enumerate}
	\item
		\begin{itemize}
			\item 
		\end{itemize}
\end{enumerate}

\paragraph*{Загальні вимоги до програми}
\begin{enumerate}
	\item Програма повинна бути незалежна від даних. Вхідний файл має
		обиратися в діалозі з користувачем. Передбачається, що вхідні дані знаходяться
		в текстовому файлі, обсяг даних не відомий. Потрібно забезпечити можливість
		модифікації та збереження даних.
	\item Слід уможливити перетворення даних (логарифмування,
		стандартизація, зсув).
	\item Після перетворення або вилучення аномальних значень користувач
		повинен мати можливість повернутися до початкових даних.
	\item Необхідно нанести на одну площину з гістограмою графік статистичної
		функції щільності, а на площину з графіком емпіричної функції розподілу ---
		графік статистичної функції розподілу разом із її довірчими інтервалами.
	\item Результатом використання критерію згоди повинні бути як проміжні
		результати (статистика критерію та її критичне значення), так і висновок (чи є
		відтворення розподілу достовірне).
	\item Результати виконання всіх обчислень мають виводитись у вигляді
		таблиць, графіків і текстових коментарів.
	\item Для кожного графіка слід виконати автоматичне маштабування,
		зобразити шкалу й показати одиниці виміру.
	\item Відображення результатів повинне відповідати точності обчислень.
\end{enumerate}

\AfterChapter{}
\end{document}
